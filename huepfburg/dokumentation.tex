\documentclass[a4paper,10pt,ngerman]{scrartcl}
\usepackage{babel}
\usepackage{hyperref}
\usepackage[T1]{fontenc}
\usepackage[utf8x]{inputenc}
\usepackage[a4paper,margin=2.5cm,footskip=0.5cm]{geometry}

\newcommand{\Aufgabe}{Aufgabe 5: Hüpfburg} % Aufgabennummer und Aufgabennamen angeben
\newcommand{\TeamId}{00730}                       % Team-ID aus dem PMS angeben
\newcommand{\TeamName}{SilverBean}                 % Team-Namen angeben
\newcommand{\Namen}{Philip Gilde}           % Namen der Bearbeiter/-innen dieser Aufgabe angeben

\usepackage{scrlayer-scrpage, lastpage}
\setkomafont{pageheadfoot}{\large\textrm}
\lohead{\Aufgabe}
\rohead{Team-ID: \TeamId}
\cfoot*{\thepage{}/\pageref{LastPage}}

% Position des Titels
\usepackage{titling}
\setlength{\droptitle}{-1.0cm}

% Für mathematische Befehle und Symbole
\usepackage{amsmath}
\DeclareMathOperator*{\argmax}{arg\,max}
\usepackage{amssymb}

% Für Bilder
\usepackage{graphicx}
\graphicspath{ {./bilder/}}
% Für Algorithmen
\usepackage{algpseudocode}
% Für Quelltext
\usepackage{listings}
\lstset{literate=%
  {Ö}{{\"O}}1
{Ä}{{\"A}}1
{Ü}{{\"U}}1
{ß}{{\ss}}1
{ü}{{\"u}}1
{ä}{{\"a}}1
{ö}{{\"o}}1
}
\usepackage{color}
\definecolor{mygreen}{rgb}{0,0.6,0}
\definecolor{mygray}{rgb}{0.5,0.5,0.5}
\definecolor{mymauve}{rgb}{0.58,0,0.82}
\lstset{
  keywordstyle=\color{blue},commentstyle=\color{mygreen},
  stringstyle=\color{mymauve},rulecolor=\color{black},
  basicstyle=\footnotesize\ttfamily,numberstyle=\tiny\color{mygray},
  captionpos=b, % sets the caption-position to bottom
  keepspaces=true, % keeps spaces in text
  numbers=left, numbersep=5pt, showspaces=false,showstringspaces=true,
  showtabs=false, stepnumber=2, tabsize=2, title=\lstname
}
\lstdefinelanguage{JavaScript}{ % JavaScript ist als einzige Sprache noch nicht vordefiniert
  keywords={break, case, catch, continue, debugger, default, delete, do, else, finally, for, function, if, in, instanceof, new, return, switch, this, throw, try, typeof, var, void, while, with},
  morecomment=[l]{//},
  morecomment=[s]{/*}{*/},
  morestring=[b]',
  morestring=[b]",
  sensitive=true
}
\usepackage{csquotes}
\newcommand\mdoubleplus{\mathbin{+\mkern-10mu+}}
% Diese beiden Pakete müssen zuletzt geladen werden
%\usepackage{hyperref} % Anklickbare Links im Dokument
\usepackage{cleveref}

% Daten für die Titelseite
\title{\textbf{\Huge\Aufgabe}}
\author{\LARGE Team-ID: \LARGE \TeamId \\\\
  \LARGE Team-Name: \LARGE \TeamName \\\\
  \LARGE Bearbeiter/-innen dieser Aufgabe: \\
  \LARGE \Namen\\\\}
\date{\LARGE\today}

\begin{document}

\maketitle
\tableofcontents

\vspace{0.5cm}

\section{Zusammenfassung}
Die Aufgabe wird mit Hilfe einer Breitensuche gelöst, welche für jeden
Spielschritt alle Felder, auf denen Mika und Sasha zu diesem sein können,
findet und miteinander vergleicht, bis eine Abbruchbedingung erfüllt ist.
\section{Lösungsidee}
Das Spielfeld mit $n$ Feldern wird als gerichteter Graf mit den Knoten
$V=\{1,2,\ldots,n\}$ und den Kanten $E \subseteq V \times V$ als Menge von
2-Tupeln der Ein- und Ausgangsknoten der Kanten modelliert, in dem jedes Feld
für einen Knoten und jeder Pfeil für eine Kante steht. \\ Die Spieler starten
an den Knoten $v_1$ und $v_2$. Die möglichen Positionen beim ersten
Spielschritt $t=0$ sind für Spieler 1 $M_{1,0}=\{v_1\}$ und für Spieler 2
$M_{2,0}=\{v_2\}$. Für jeden weiteren Spielschritt werden von allen Positionen
des vorherigen Schrittes alle ausgehenden Kanten zum nächsten Knoten
traversiert:
\begin{align*}
  M_{s,t+1} = \{w | (v, w) \in E | v \in M_{s, t}\}
\end{align*}
Dieser Prozess wird nun wiederholt, bis sich die Knoten, auf denen die Spieler zu einem Zeitpunkt sein können, überlappen;
formal bis $M_{1, t} \cap M_{2, t} \neq \emptyset$. Wenn die beiden Spieler sich allerdings nie treffen können, würde man so
unendlich lange wiederholen. Um das zu verhindern, wird weiterhin eine Menge $B$ aller besuchten Knotenpaare, zu denen die
beiden Spielern schon gleichzeitig sein könnten geführt:
\begin{align*}
  B_0     & = \emptyset                                \\
  B_{t+1} & = B_{t} \cup (M_{1,t+1} \times M_{2, t+1})
\end{align*}
Die Wiederholung wird abgebrochen, wenn in einem Schritt keine weiteren Knotenpaare besucht werden: $B_t = B_{t+1}$.
Das hat folgenden Grund: Wenn die Spieler sich in einem Schritt $t$ treffen, müssen sie beide von zwei verschiedenen
Knoten dort hin gekommen sein; zu diesem Paar müssen die Spieler auch durch ein Paar unterschiedlicher Knoten gekommen sein,
und so weiter. Wenn in einem Schritt allerdings keine neuen solchen Knotenpaare betreten werden, heißt das, dass keine neuen
Pfade entdeckt werden können, die zu einem Knoten führen könnten, diese wurden alle schon in vorherigen Schritten betreten und
davon ausgehende Pfade erkundet. Wenn die ausgehenden Pfade nämlich nicht vollständig erkundet wurden, würden durch diese neue
Knotenpaare hinzukommen. So terminiert das Verfahren in höchstens $|V|^2 - |V|$ Schritten, da es höchstens so viele ungleiche Knotenpaare gibt. \\
Um, wenn das Spiel lösbar ist, den Weg zu rekonstruieren, wird weiterhin eine  Zuordnung $H_t:B_t \rightarrow B_{t-1}$ eingeführt,
die für erkundete Knotenpaare die jeweiligen Herkunftsknoten speichert. Es kann auch mehrere Herkunftsknotenpaare geben, allerdings wird für
jedes Knotenpaar nur ein Herkunftsknotenpaar benötigt, um den Pfad dahin zurückzuverfolgen. 
Dabei wird immer der kürzeste Pfad zum Knotenpaar verwendet, in dem, sobald einmal ein Pfad gefunden wurde, dieser in späteren Iterationen nicht verändert wird. 
\begin{align*}
  H_0(1,2)         & = (-1, -1)                                                                                                        \\
  H_{t+1}(v_1,v_2) & = \begin{cases}
                         H_{t}(v_1, v_2)                                                                    & wenn (v_1,v_2) \in B_t     \\
                         (w_1, w_2)|(w_1,v_1), (w_2,v_2) \in E \land w_1 \in M_{1,t} \land w_2 \in M_{2, t} & wenn (v_1,v_2) \in B_{t+1}
                       \end{cases}
\end{align*}\\\\
Wenn sich die Spieler im Schritt $t$ auf dem Knoten $v$ treffen, kann man über $H_{t}(v,v)$ die Herkunftsknoten der
beiden Spieler bestimmen. Für das Herkunftsknotenpaar kann man genauso die Herkunftsknoten bestimmen und so weiter
bis die Herkunftsknoten irgendwann (1,2) sind.\\\\
Aus den Gleichungen lässt sich mithilfe der Abbruchbedingungen ein Algorithmus formulieren, der die Werte von $M$, $B$
und $H$ im jeweils nächsten Schritt, ausgehend von den Startwerten solange berechnet, bis eine der Abbruchbedingungen erfüllt ist:
\begin{algorithmic}
  \State$M_1 \gets \{1\}$
  \State$M_2 \gets \{2\}$
  \State$M_{1l} \gets \emptyset$
  \State$M_{2l} \gets \emptyset$
  \State$B \gets \{(1, 2)\}$
  \State$B_l \gets \emptyset$
  \State$H \gets (1, 2) \mapsto (-1, -1)$
  \While{$M_1 \cap M_2 = \emptyset \land B \neq B_l$}
  \State$B_l, M_{1l}, M_{2l} \gets B, M_1, M_2$
  \State$M_1 \gets \{w | (v, w) \in E | v \in M_1\} $
  \State$M_2 \gets \{w | (v, w) \in E | v \in M_2\} $
  \State$B \gets B \cup M_1 \times M_2$
  \State$H \gets (v1, v2) \mapsto \begin{cases}
      H(v_1, v_2)                                                                 & wenn (v_1,v_2) \in B_l \\
      (w_1, w_2)|(w_1,v_1), (w_2,v_2) \in E \land w_1 \in M_1l \land w_2 \in M_2l & wenn (v_1,v_2) \in B
    \end{cases}$
  \EndWhile
  \If{$M_1 \cap M_2 \neq \emptyset$}
  \State$P \gets [(x,x) | x \in M_1 \cap M_2]$
  \While{$P[-1] \neq (1, 2)$}
  \State$P \gets P +\!\!\!+ [H(P[-1])]$
  \EndWhile
  \State \Return $P$
  \Else
  \State \Return $[]$
  \EndIf
\end{algorithmic}
\section{Umsetzung}
Die Lösung der Aufgabe wurde in Python 3.11 implementiert. Dabei wurde der
\lstinline|set|-Datentyp für die endlichen Mengen und der
\lstinline|dict|-Datentyp für die Zuordnung $H$ verwendet. Die Menge der Kanten
wird als Menge von \lstinline|tupel|s dargestellt. Die Knotenpaar-Tupel werden
auch mit dem Datentyp \lstinline|tupel| dargestellt. Das Programm ist eine
direkte Transkription des Pseudocodes aus der Lösungsidee.\\
Wenn das Programm gestartet wurde, muss der Benutzer den Pfad zur Eingabedatei angeben, dann gibt das Pogramm entweder aus, wie die Spieler hüpfen müssen um sich zu treffen, oder es gibt aus, dass das Spiel nicht lösbar ist.\\

\section{Beispiele}
Die Beispiele stammen von der BwInf-Webseite und sind im Verzeichnis \lstinline|beispiele| zu finden.\\
huepfburg0.txt: \\
\begin{lstlisting}
  Treffen möglich
  Pfad 1:
   1 -> 18 -> 13 -> 10
  Pfad 2:
   2 -> 19 -> 20 -> 10
\end{lstlisting}
huepfburg1.txt: \\
\begin{lstlisting}
  Treffen möglich
  Pfad 1:
   1 -> 4 -> 5 -> 6 -> 7 -> 8 -> 9 -> 10 -> 11 -> 12 -> 13 -> 14 -> 15 ->
    16 -> 17 -> 1 -> 4 -> 5 -> 6 -> 7 -> 8 -> 9 -> 10 -> 11 -> 12 -> 13 
    -> 14 -> 15 -> 16 -> 17 -> 1 -> 4 -> 5 -> 
  6 -> 7 -> 8 -> 9 -> 10 -> 11 -> 12 -> 13 -> 14 -> 15 -> 16 -> 17 -> 1 
  -> 4 -> 5 -> 6 -> 7 -> 8 -> 9 -> 10 -> 11 -> 12 -> 13 -> 14 -> 15 -> 16
   -> 17 -> 1 -> 4 -> 5 -> 6 -> 7 -> 8 -> 9 -> 10 -> 11 -> 12 -> 13 -> 14
    -> 15 -> 16 -> 17 -> 1 -> 4 -> 5 -> 6 -> 7 -> 8 -> 9 -> 10 -> 11 -> 
    12 -> 13 -> 14 -> 15 -> 16 -> 17 -> 1 -> 4 -> 5 -> 6 -> 7 -> 8 -> 9 
    -> 10 -> 11 -> 
  12 -> 13 -> 14 -> 15 -> 16 -> 17 -> 1 -> 4 -> 5 -> 6 -> 7 -> 8 -> 9 -> 
  10 -> 11 -> 12 -> 13 -> 14 -> 15 -> 16 -> 17 -> 1 -> 4
  Pfad 2:
   2 -> 3 -> 4 -> 5 -> 6 -> 7 -> 8 -> 9 -> 10 -> 11 -> 12 -> 13 -> 14 -> 
   15 -> 16 -> 17 -> 1 -> 2 -> 3 -> 4 -> 5 -> 6 -> 7 -> 8 -> 9 -> 10 -> 
   11 -> 12 -> 13 -> 14 -> 15 -> 16 -> 17 -> 
  1 -> 2 -> 3 -> 4 -> 5 -> 6 -> 7 -> 8 -> 9 -> 10 -> 11 -> 12 -> 13 -> 14
   -> 15 -> 16 -> 17 -> 1 -> 2 -> 3 -> 4 -> 5 -> 6 -> 7 -> 8 -> 9 -> 10 
   -> 11 -> 12 -> 13 -> 14 -> 15 -> 16 -> 17 -> 1 -> 2 -> 3 -> 4 -> 5 -> 
   6 -> 7 -> 8 -> 9 -> 10 -> 11 -> 12 -> 13 -> 14 -> 15 -> 16 -> 17 -> 1 
   -> 2 -> 3 -> 4 -> 5 -> 6 -> 7 -> 8 -> 9 -> 10 -> 11 -> 12 -> 13 -> 14 
   -> 15 -> 16 -> 17 -> 1 -> 2 -> 3 -> 4 -> 5 -> 6 -> 7 -> 8 -> 9 -> 10 
   -> 11 -> 12 -> 13 -> 14 -> 15 -> 16 -> 17 -> 1 -> 2 -> 3 -> 4
\end{lstlisting}
huepfburg2.txt: \\
\begin{lstlisting}
Treffen möglich
Pfad 1:
  1 -> 51 -> 76 -> 59 -> 110 -> 116 -> 112 -> 95 -> 51
Pfad 2:
  2 -> 106 -> 136 -> 108 -> 100 -> 12 -> 114 -> 3 -> 51
\end{lstlisting}
huepfburg3.txt: \\
\begin{lstlisting}
Treffer nicht möglich
\end{lstlisting}
huepfburg4.txt: \\
\begin{lstlisting}
  Treffen möglich
  Pfad 1:
   1 -> 99 -> 89 -> 88 -> 78 -> 77 -> 76 -> 66 -> 56 -> 55 -> 54 -> 44 ->
    43 -> 33 -> 23 -> 13 -> 12
  Pfad 2:
   2 -> 12 -> 11 -> 100 -> 12 -> 11 -> 100 -> 2 -> 12 -> 11 -> 100 -> 2 
   -> 12 -> 11 -> 100 -> 2 -> 12
\end{lstlisting}	
\section{Quellcode}
\lstinputlisting[language=python]{huepfburg doc.py}
\end{document}